\documentclass[10pt, a4paper]{article}
\usepackage[utf8]{inputenc}
\usepackage{makeidx}
\usepackage[scale=0.85]{sourcecodepro}
\usepackage{amsmath, amssymb, amsfonts}
\usepackage{microtype}
\usepackage{bookmark}
\usepackage{geometry}
\usepackage[print=m, lang=cpp]{fortex}

\makeindex

\usemintedstyle{ayu}
\setfortex{tabsize=2, breaksymbol=\scriptsize{$\hookrightarrow$}, fontsize=\small}
\DefineShortVerb{\"}

\begin{document}

This file is just a controller for the Fortran that I wrote earlier. 

I initially tried doing this in "C++", but I was unable to extract the "TTree" from the "TFile". 
There is obviously a way to do it, the the ROOT docs are kinda horrific. 

At some point I do want to translate this to regular "C++" but idk when that will be. 
\index{C++}
\begin{code}
#include <stdio.h>
#include <stdlib.h>
#include <math.h>
#include <TFile.h>
#include <TTreeReader.h>
#include <TTreeReaderArray.h>
#include <TCanvas.h>
#include <TGraph.h>
#include "conv.h"
\end{code}

We need to declare a header file for "find_circ". 
This is not actually in a header file but that should be find because I am only going to "#include" it here anyway.

We also need to use \verb|extern "C"| to avoid the "C++" name mangling.

This should probs belong in a header file, but for the moment because fortex is not smart enough to have subfiles yet, ill just put it here manually. 

\begin{code}
int main(void) {
	init_conv();
	TFile file("../ns~real.root");
// 	TTree *tree = (TTree *) file.Get("tuple/tuple");
	TTree *tree = (TTree *) file.Get("RICH/OfflineMonopoleFinderR1Gas/OfflineMonopoleFinderR1Gas");
	tree->SetBranchStatus("*", 0);
	tree->SetBranchStatus("nHits", 1);
// 	tree->SetBranchStatus("HitRICH", 1);
	tree->SetBranchStatus("HitPosX", 1);
	tree->SetBranchStatus("HitPosY", 1);
	tree->SetBranchStatus("HitPosZ", 1);
// 	tree->SetBranchStatus("Generated_PE", 1);
// 	tree->SetBranchStatus("Generated_PX", 1);
// 	tree->SetBranchStatus("Generated_PY", 1);
// 	tree->SetBranchStatus("Generated_PZ", 1);
\end{code}
we need to allocate the arrays we are going to read the leaf elements into. 
However, the size of those arrays is contained as a leaf element ($N_\text{hit}$). 

This is obviously problematic, as $N_\text{hit}$ is not a constant for each element. 

Obviously this is not idea, but for the moment it is both fast to code and execute than it is to do "realloc" on every element. 
Besides as we are passing the length of the array as an explicit element to the Fortran side of the code it is a non-issue, as we just read that. 

\begin{code}
	int N_max = tree->GetLeaf("nHits")->GetMaximum();
\end{code}

Using this information we can then allocate the rest of the arrays. 

\begin{code}
	int N;
// 	float sensor[N_max];
	float x[N_max];
	float y[N_max];
	float z[N_max];
	
// 	double pe;
// 	double px;
// 	double py;
// 	double pz;
	
	float retval[4];
\end{code}

We then use pointers to $x$, $y$ and $N$ that get loaded every time "tree->GetEntry(i)" is called. 
This looks more parallisable than the "TTreeReader" implementation, but it still relys on these global and that makes it annoying. 

\begin{code}
	tree->SetBranchAddress("nHits", &N);
// 	tree->SetBranchAddress("HitRICH", &sensor);
	tree->SetBranchAddress("HitPosX", &x);
	tree->SetBranchAddress("HitPosY", &y);
	tree->SetBranchAddress("HitPosZ", &z);
// 	tree->SetBranchAddress("Generated_PE", &pe);
// 	tree->SetBranchAddress("Generated_PX", &px);
// 	tree->SetBranchAddress("Generated_PY", &py);
// 	tree->SetBranchAddress("Generated_PZ", &pz);
\end{code}

Here we are going to be calling a loop to compute our FFTs. 
We do this in Fortran because it is a lot easier to work with arrays in Fortran than it is a C based language.

Ideally I want to parallelise this look, but for the moment single threaded code seems to be good enough.

\begin{code}
	int n_entry = tree->GetEntries();
	printf("Number of events: %d\n\n", n_entry);

	for (int i=0; i<n_entry; i++) {
		tree->GetEntry(i);

		if (N == 0) {
			continue;
		}
		
		find_circ_centre(&N, x, y, z, retval);
		printf("x: %g, y: %g, r: %g, mag: %g\n", retval[0], retval[1], retval[2], retval[3]/(128 * 128 * 16 * 1000));
\end{code}

\begin{code}
	}
	delete_conv();
	return 0;
}
\end{code}


\printindex
\end{document}
